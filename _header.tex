\documentclass[titlepage]{article}

\usepackage{listings}
\usepackage[dvipsnames,table]{xcolor}
\usepackage[firstpage]{draftwatermark}
\usepackage{eso-pic}
\usepackage[T1]{fontenc}
\usepackage[french]{babel}
\usepackage[utf8]{inputenc}
\usepackage[hidelinks]{hyperref}
% \usepackage[hyperpageref]{backref}
% \usepackage{chngcntr}
\usepackage{amsmath}
\usepackage{amssymb}
\usepackage{geometry} % Règle la marge en haut/bas/droite/gauche
\usepackage{times} % Change la police en Times new roman
\usepackage{etoolbox}
\usepackage{titlesec}
\usepackage{tocloft}
\usepackage{multirow}
\usepackage{tikz} % Pour faire des figures 3D
\usepackage{subcaption}

\SetWatermarkLightness{0.8}
\SetWatermarkAngle{50}
\SetWatermarkScale{1.4}
\SetWatermarkFontSize{3cm}
\SetWatermarkText{Confidentiel}

\usetikzlibrary{patterns,perspective}
\usetikzlibrary{decorations.pathreplacing,calligraphy}

% --- Permet de régler l'espace entre le nombre et le titre dans toc ---

\setlength\cftsubsecnumwidth{1em}
\setlength\cftsubsubsecnumwidth{1.8em}
\setlength\cftparanumwidth{2.6em}

\setlength\cftsubsubsecindent{2.6em}
\setlength\cftparaindent{4.4em}

\setlength\cftfignumwidth{2.8em}

% --- Changement nombre figure --- (package chngcntr)

\counterwithin{figure}{section}

% --- Réglages du package french pour que les tirets soient des points dans les listes ---

%\frenchbsetup{StandardLists=true}

% --- Règle les marges ---

\geometry{hmargin=2.5cm,vmargin=2.5cm}

% --- glossary ---

\usepackage[toc,acronym]{glossaries}
\makeglossaries

% --- Nomenclature ---

\usepackage{nomencl}
\makenomenclature

\renewcommand\nomgroup[1]{%
  \item[\bfseries
  \ifstrequal{#1}{A}{Lettres latines}{%
  \ifstrequal{#1}{B}{Lettres grecques}{%
  \ifstrequal{#1}{C}{Indices}{%
  \ifstrequal{#1}{D}{Exposants}{%
  \ifstrequal{#1}{E}{Nombres sans dimension}{}}}}}%
]}

\newcommand{\nomunit}[1]{%
\renewcommand{\nomentryend}{\hspace*{\fill}#1}}

% --- Pour mettre les logos de l'ISIMA et A&D sur la page titre ---
 
\newcommand{\HRule}{\rule{\linewidth}{0.5mm}}
\newcommand{\blap}[1]{\vbox to 0pt{#1\vss}}
\newcommand\AtUpperLeftCorner[3]{%
  \put(\LenToUnit{#1},\LenToUnit{\dimexpr\paperheight-#2}){\blap{#3}}%
}
\newcommand\AtUpperRightCorner[3]{%
  \put(\LenToUnit{\dimexpr\paperwidth-#1},\LenToUnit{\dimexpr\paperheight-#2}){\blap{\llap{#3}}}%
}

% --- Section et autres ---

\setcounter{tocdepth}{4}
\setcounter{secnumdepth}{4}

\newcommand{\wsection}[1]{
\section*{#1}
\addcontentsline{toc}{section}{#1}
}
\newcommand{\wsubsection}[1]{
\subsection*{#1}
\addcontentsline{toc}{subsection}{#1}
}

\titleformat*{\section}{\huge\bfseries}
\titleformat*{\subsection}{\LARGE\bfseries}
\titleformat*{\subsubsection}{\Large\bfseries}
\titleformat*{\paragraph}{\large\bfseries}
\titleformat*{\subparagraph}{\large\bfseries}

\renewcommand{\thesection}{\Roman{section}}
\renewcommand{\thesubsection}{\arabic{subsection}}
\renewcommand{\thesubsubsection}{\arabic{subsection}.\arabic{subsubsection}}
\renewcommand{\theparagraph}{\arabic{subsection}.\arabic{subsubsection}.\arabic{paragraph}}

% --- ref ---

\newcommand{\completeref}[1]{\autoref{#1}, page \pageref{#1}}
\newcommand{\bracompref}[1]{\autoref{#1} (page \pageref{#1})}

\newcommand{\Iref}[1]{\cRM{1}.\ref{#1}}
\newcommand{\IIref}[1]{\cRM{2}.\ref{#1}}
\newcommand{\IIIref}[1]{\cRM{3}.\ref{#1}}

% --- forge ---

\newcommand{\forge}{FORGE$^\circledR$}
\newcommand{\forgews}{FORGE$^\circledR$ }

% --- chiffre romains ---

% Romain
\newcommand{\cRM}[1]{\MakeUppercase{\romannumeral #1}} % Capital
\newcommand{\cRm}[1]{\textsc{\romannumeral #1}} % Petit majuscule
\newcommand{\crm}[1]{\romannumeral #1}
% Siècle %
\newcommand{\siecle}[1]{\cRM{#1}\textsuperscript{e}~siècle}

% --- listings ---

\definecolor{codegreen}{rgb}{0,0.6,0}
\definecolor{codegray}{rgb}{0.5,0.5,0.5}
\definecolor{codepurple}{rgb}{0.58,0,0.82}
\definecolor{backcolour}{rgb}{0.95,0.95,0.92}

\lstdefinestyle{mystyle}{
    backgroundcolor=\color{backcolour},   
    commentstyle=\color{codegreen},
    keywordstyle=\color{magenta},
    numberstyle=\tiny\color{codegray},
    stringstyle=\color{codepurple},
    basicstyle=\ttfamily\footnotesize,
    breakatwhitespace=false,         
    breaklines=true,                 
    captionpos=b,                    
    keepspaces=true,                 
    numbers=left,                    
    numbersep=5pt,                  
    showspaces=false,                
    showstringspaces=false,
    showtabs=false,                  
    tabsize=2
}

\definecolor{classclr}{RGB}{71,196,160}
\definecolor{funcclr}{RGB}{255,219,87}

\newcommand{\class}[1]{\textcolor{Emerald}{#1}}
\newcommand{\widget}[1]{\textcolor{OrangeRed}{#1}}
\newcommand{\function}[1]{\textcolor{Orange}{#1}}
\newcommand{\variable}[1]{\textcolor{Cyan}{#1}}

\lstset{style=mystyle}

\renewcommand\lstlistingname{\'{E}lément de code}
\renewcommand\lstlistlistingname{Table des éléments de code}
\def\lstlistingautorefname{élément de code}